\documentclass{article}

\usepackage{graphicx,color,subcaption}
\usepackage[letterpaper, margin=1in]{geometry}
\usepackage{setspace}  % use this package to set line spacing as desired
\usepackage{times}  % set Times New Roman as the font
\usepackage[explicit]{titlesec}  % title control and formatting
\usepackage[titles]{tocloft}  % table of contents control and formatting
%\usepackage[backend=biber, bibstyle=ieee]{biblatex}
\usepackage[bookmarks=true, hidelinks]{hyperref}
\setlength {\marginparwidth }{2cm}
\usepackage{todonotes}
\usepackage{amssymb,amsmath,amsthm,mathabx}
\usepackage{algorithm}
\usepackage{algpseudocode}
\usepackage{tikz}
\usepackage{nicematrix}
\usepackage{letltxmacro}
\usepackage[makeroom]{cancel}
\usepackage{upgreek}
\usepackage{csquotes}
\usepackage{mathbbol}
\usetikzlibrary{decorations.pathreplacing}
\usepackage{amsfonts}
\usepackage{soul}
\usepackage[page]{appendix}  % for appendices
\usepackage{array}
\usepackage{enumitem}
\usepackage[normalem]{ulem}
\usepackage{fancyhdr}
\usepackage[most]{tcolorbox}
\usepackage{listings}
\usepackage{breqn}
\usepackage{pgfplotstable}
\usepackage{booktabs}
\usepackage{longtable}
\usepackage{csvsimple}
\usepackage{bm}
\usepackage{float}

\definecolor{codegreen}{rgb}{0,0.6,0}
\definecolor{codegray}{rgb}{0.5,0.5,0.5}
\definecolor{codepurple}{rgb}{0.58,0,0.82}
\definecolor{backcolour}{rgb}{0.95,0.95,0.92}

% Define first and second derivative notation
\newcommand{\p}{^{\prime}}
\newcommand{\pp}{^{\prime\prime}}

\newcommand{\R}{\mathbb{R}}
\newcommand{\C}{\mathbb{C}}

\newcommand{\mcO}{\mathcal{O}}

\newcommand{\BE}[1]{\bm{\mathsf{#1}}}

\newcommand{\ip}[2]{\left\langle #1, #2 \right\rangle}

\DeclareMathOperator{\diag}{diag}
\DeclareMathOperator{\spanvec}{span}

\newcommand{\deriv}[3][]{\frac{\mathrm{d}^{#1}#2}{\mathrm{d}#3^{#1}}}
\newcommand{\pderiv}[3][]{\frac{\partial^{#1}#2}{\partial#3^{#1}}}

\lstdefinestyle{mystyle}{
    backgroundcolor=\color{backcolour},   
    commentstyle=\color{codegreen},
    keywordstyle=\color{magenta},
    numberstyle=\tiny\color{codegray},
    stringstyle=\color{codepurple},
    basicstyle=\ttfamily\footnotesize,
    breakatwhitespace=false,         
    breaklines=true,                 
    captionpos=b,                    
    keepspaces=true,                 
    numbers=left,                    
    numbersep=5pt,                  
    showspaces=false,                
    showstringspaces=false,
    showtabs=false,                  
    tabsize=4
}

\lstset{style=mystyle}

\begin{document}

    \noindent We want to identify the Fréchet derivative of the dispersion relation $\kappa(\omega)$ (or $\omega(\kappa)$). We know
    \begin{align}
        -(\Delta + \omega^2)\psi &= 0, \quad x\in\Omega \\
        \psi_n &= 0,\quad x\in\Gamma
    \end{align}
    where $\psi$ is a Laplace eigenfunction corresponding to $\lambda = \omega^2$ can (or will?) lead to a trapped mode on a periodic function. Then, the quasi-periodicity condition
    \begin{align}
        \psi(\vec{x}+n\vec{d}) &= e^{in\kappa d}\psi(x)
    \end{align}
    links $\kappa$ and $\omega$. We define a function, $F: \Gamma \to \kappa$, that is numerically available from Riley's forward solver. We want to find the Fréchet derivative of $F$ with respect to the parameterization of the boundary $\Gamma$, $\bar{\gamma}$. This is done by making a collection of points and using some interpolation method to construct the boundary (polygon, splines, or polynomial interpolation perhaps). In order to achieve this, we differentiate with respect to $\tau$ which parametrizes the change of the boundary using the product rule,
    \begin{align}\label{eq:Frechet derivative relation}
        \psi_\tau(\vec{x}+n\vec{d}) &= in\kappa' de^{in\kappa d}\psi(x) + e^{in\kappa d}\psi_\tau(\vec{x})
    \end{align}
    where $\kappa'$ is the quantity of interest. We now need to identify an expression for $\psi'$ from the Grinfeld paper. The eigenfunction variation, $\psi_\tau$, in $\Omega$ is governed by the system 
    \begin{align}\label{eq:psi_tau defn}
        \Delta \psi_\tau + \lambda \psi_\tau &= -\lambda'\psi \\
        \intertext{with the normalization condition}
        2\int_\Omega \psi_\tau \psi d\Omega + \int_S C\psi^2ds &= 0.
        \intertext{The boundary condition for the Neumann case is}
        \frac{\partial \psi_\tau}{\partial n}\bigg|_S &= \nabla_SC\cdot \nabla_S\psi + \lambda C \psi + C\Delta_S\psi
    \end{align}
    \todo{How important is the normalization condition?}
    where $\nabla_S := \nabla - \vec{n}(\vec{n}\cdot\nabla)$ is the surface gradient operator that produces a gradient tangent to the surface, $\Delta_S$ is the Laplace-Beltrami operator on the surface \todo{I think that in this case $\Delta_S$ simplifies to $\Delta$, right?} and $\lambda' = \int_S C\left( -\lambda \psi^2 + |\nabla_S \psi|^2 \right) dS$. C is the rate of displacement of the surface $S_\tau$ in the instantaneously normal direction, $C = \lim_{h\to0^+}\frac{(P^*-P)\cdot \vec{n}}{h}$ where $P^*$ is evaluated at $\tau+h$ and $P$ evaluated at $\tau$.
    % \begin{center}
    %     \includegraphics[width=0.4\textwidth]{../Figures/Grinfeld_C_figure.png} 
    % \end{center}

    \todo[inline]{Does this mean that we would need to solve the PDE for $\psi_\tau$ in \eqref{eq:psi_tau defn}?}

    \begin{algorithm}
    \caption{Pseudocode for construction of $\kappa'$}
    \begin{algorithmic}[1] 
        \State \textbf{Input:} $\Gamma$, $\omega^2$
        \State \textbf{Output:} $\kappa'$
        \State Generate the eigenfunction $\psi$ using Riley's forward solver
        \State For an initial guess perturbation, compute $C$, $\lambda'$.
        \State Solve the PDE in \eqref{eq:psi_tau defn} for $\psi_\tau$ using above BCs.
        \State Algebraically solve for $\kappa'$ using the relation \eqref{eq:Frechet derivative relation}
        \State Use $\kappa'$ in a Newton step to update parameterization of $\Gamma$.
        \State Repeat until converged.
    \end{algorithmic}
    \end{algorithm}
    \todo[inline]{I am assuming that we will need some intitial guess direction to take, not quite sure how we would get the first $\kappa'$ otherwise as $C$ would be 0. Maybe this will become more clear as I look into the GitHub repos you shared tomorrow.}
    
\end{document}